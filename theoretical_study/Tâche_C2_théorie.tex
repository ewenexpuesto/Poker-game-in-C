\documentclass{article}
\usepackage[utf8]{inputenc}
\usepackage{amsmath}
\usepackage{amssymb}
\usepackage[a4paper, margin=1.5cm]{geometry}

\title{Tâche C.2}
\date{}  % No date
\author{}  % No author

\begin{document}

\maketitle

En notant $\Omega = \{1, 1, 2, 2, 3, 3\}$ les cartes du jeu. L'ensemble des couples qu'une joueuse peut tirer est donc $\{(1,1), (1,2), (1,3), (2,2), (2,3), (3,3)\}$. En modélisant le tirage d’une carte par une loi uniforme, et en supposant que chaque tirage est indépendant du suivant, la probabilité de tirer chaque couple est

\begin{itemize}
    \item $P(1,1) = \frac{2}{6} \times \frac{1}{5} = \frac{1}{15}$
    \item $P(1,2) = \frac{2}{6} \times \frac{2}{5} \times 2 = \frac{4}{15}$
    \item $P(1,3) = \frac{2}{6} \times \frac{2}{5} \times 2 = \frac{4}{15}$
    \item $P(2,2) = \frac{2}{6} \times \frac{1}{5} = \frac{1}{15}$
    \item $P(2,3) = \frac{2}{6} \times \frac{2}{5} \times 2 = \frac{4}{15}$
    \item $P(3,3) = \frac{2}{6} \times \frac{1}{5} = \frac{1}{15}$
\end{itemize}

\end{document}
